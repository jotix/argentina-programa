% Created 2023-05-30 Tue 21:49
% Intended LaTeX compiler: pdflatex
\documentclass[11pt]{article}
\usepackage[utf8]{inputenc}
\usepackage[T1]{fontenc}
\usepackage{graphicx}
\usepackage{longtable}
\usepackage{wrapfig}
\usepackage{rotating}
\usepackage[normalem]{ulem}
\usepackage{amsmath}
\usepackage{amssymb}
\usepackage{capt-of}
\usepackage{hyperref}
\date{\today}
\title{}
\hypersetup{
 pdfauthor={},
 pdftitle={},
 pdfkeywords={},
 pdfsubject={},
 pdfcreator={Emacs 28.2 (Org mode 9.5.5)}, 
 pdflang={English}}
\begin{document}

\tableofcontents

\section{Ejercicio 1:}
\label{sec:org2011be5}
Escriba un programa que permita que el usuario ingrese un
número entero y luego lo imprima por pantalla
\begin{verbatim}
numero_entero = int(input("ingrese un número entero"))
print("El número ingresado es:", numero_entero)
\end{verbatim}
\section{Ejercicio 2:}
\label{sec:orgd9ae5cc}
Escriba un programa que permita que el usuario ingrese un
número flotante y luego lo imprima por pantalla.
\begin{verbatim}
numero = float(input("ingrese un número"))
print("El número ingresado es:", numero)
\end{verbatim}
\section{Ejercicio 3:}
\label{sec:org22b7e39}
Escriba un programa que permita que el usuario ingrese dos
números e imprima por pantalla la suma, resta y multiplicación de
dichos números.
\begin{verbatim}
numero1 = input("Ingrese un número:")
numero2 = input("Ingrese otro número:")
print("La suma es:", numero1 + numero2)
print("La resta es:", numero1 - numero2)
print("la multiplicación es", numero1 * numero2)
\end{verbatim}
\section{ejercicio 4:}
\label{sec:org2e3e49c}
Escriba un programa que permita calcular el área y perímetro
de un triángulo.
\begin{verbatim}
### suponemos que es un triangulo equilatero
base = input("ingrese la base del triángulo:")
altura = input("Ingrese la altura del triángulo:")
print("El area del triángulo es:", base * altura / 2)
print("El perimetro del triángulo es:", base * 3)
\end{verbatim}
\section{ejercicio 5:}
\label{sec:orgce939f8}
Escriba un programa que permita calcular el área y perímetro
de un cuadrado.
\begin{verbatim}
lado = int(input("Ingrese lado del cuadrado:"))
print("El perímetro del cuadrado es:", lado * 4)
print("El área del cuadrado es:", lado * lado)
\end{verbatim}
\section{ejercicio 6:}
\label{sec:orgf6b44b0}
Escriba un programa que permita calcular el promedio de cinco
números ingresados por el usuario
\begin{verbatim}
numero1 = input("ingrese el número 1:")
numero2 = input("ingrese el número 2:")
numero3 = input("ingrese el número 3:")
numero4 = input("ingrese el número 4:")
numero5 = input("ingrese el número 5:")
promedio = (numero1 + numero2 + numero3 + numero4 + numero5) / 5
print("El promedio de los números es:", promedio)
\end{verbatim}
\section{ejercicio 7:}
\label{sec:org60ad9a9}
Escriba un programa que imprima por pantalla “Hola Mundo!”
\begin{verbatim}
print("Hola Mundo!")
\end{verbatim}
\section{ejercicio 8:}
\label{sec:org221a617}
Escriba un programa que solicite al usuario un string y luego
imprima por pantalla dicho string.
\begin{verbatim}
string_ingresado = input("Ingrese un string:")
print("El string ingresado es:", string_ingresado)
\end{verbatim}
\section{ejercicio 9:}
\label{sec:org9b8115e}
Escriba un programa que permita almacenar un string en la
variable s y luego muestre por pantalla el contenido de s.
\begin{verbatim}
s = input("Ingrese un string:")
print("El string ingresado es:", s)
\end{verbatim}
\section{ejercicio 10:}
\label{sec:org174fc90}
Escriba un programa que permita que el usuario ingrese el
nombre de una persona y luego imprima por pantalla “Hola:” seguido
del nombre de la persona.
\begin{verbatim}
nombre = imput("Ingrese el nombre:")
print("Hola:", nombre)
\end{verbatim}
\section{ejercicio 11:}
\label{sec:orgdc003e9}
Escriba un programa que permita ingresar por teclado el nombre
de una persona, la cantidad de horas trabajadas y luego el costo de
la hora. El programa debe informar cuánto debe cobrar la persona.
\begin{verbatim}
nombre = input("Ingrese el nombre:")
horas_trabajadas = float(input("Horas trabajadas:"))
costo_hora = float(input("Costo de la hora:"))
print(nombre, "debe cobrar", horas_trabajadas * costo_hora)
\end{verbatim}
\section{ejercicio 12:}
\label{sec:orgee5faaa}
Escriba un programa que calcule el IMC (índice de Masa Corporal).
\begin{verbatim}
# (IMC = peso (kg)/ [estatura (m)]2
peso = float(input("Ingrese el peso:"))
estatura = float(input("Ingrese la estatura (en metros):"))
imc = peso / (estatura * estatura)
print("el IMC de la persona es:", imc)
\end{verbatim}
\section{ejercicio 13:}
\label{sec:orgb07ca24}
Escriba un programa que pida al usuario dos números m y
n el programa debe imprimir por pantalla el cociente y el resto de la
división de m por n. Para este ejercicio asuma que n no puede ser 0.
\begin{verbatim}
m = int(input("Ingrese el número m:"))
n = int(input("Ingrese el número m:"))
resto = m % n
cociente = m // n
print("El resto de la división de m pon es:", resto)
print("El cociente de la división de m por n es:", cociente)
\end{verbatim}
\section{ejercicio 14:}
\label{sec:orga6a28cc}
Escriba un programa que permita realizar la conversión a
dólares y euros de una cantidad de pesos ingresada por el usuario.
\begin{verbatim}
cotizacion_dolar = 100
cotizacion_euro = 120
monto_pesos = float(input("Ingrese el monto en pesos:"))
print("Equivalente en dólares:" round(monto_pesos / cotizacion_dolar, 2))
print("Equivalente en euros:" monto_pesos / cotizacion_euro)
\end{verbatim}
\section{ejercicio 15:}
\label{sec:org15654aa}
Escribir un programa que pregunte al usuario una cantidad a
invertir, el interés anual y el número de años, y muestre por pantalla el
capital obtenido en la inversión.
\begin{verbatim}
cantidad_a_invertir = float(input("Ingrese la cantidad a invertir:"))
interes_anual = float(input("Ingrese el interes anual:"))
cantidad_de_anios = int(input("Ingrese los años:"))
capital_obtenido = cantidad_a_invertir * \
    (1 + (interes_anual/100 * cantidad_de_anios)
print("Capital obtenido:",  capital_obtenido)
\end{verbatim}
\section{ejercicio 16:}
\label{sec:orgc2d6c15}
Una juguetería tiene mucho éxito en dos de sus productos:
payasos y muñecas. Suele hacer venta por correo y la empresa de logística
les cobra por el peso de cada paquete así que deben calcular el
peso de los payasos y muñecas que saldrán en cada paquete a demanda.
Cada payaso pesa 112 g y cada muñeca 75 g. Escriba un programa que
lea el número de payasos y muñecas vendidos en el último pedido y
calcule el peso total del paquete que será enviado
\begin{verbatim}
peso_payaso = 112
peso_munieca = 75
cantidad_payasos = int(input("Ingrese la cantidad de payasos:"))
cantidad_muniecas = int(input("Ingrese la cantidad de muñecas:"))
peso_total_en_kg = \
    (cantidad_payasos * peso_payaso + cantidad_muniecas * peso_munieca) / 1000
print("Peso total del pedido:", peso_total_en_kg, "kg.")
\end{verbatim}
\section{ejercicio 17:}
\label{sec:org7b024c9}
Imagine que acaba de abrir una nueva cuenta de ahorros que
ofrece el 4 \% de interés al año. Estos ahorros debido a intereses, que no
se cobran hasta finales de año, se añaden al balance final de su cuenta
de ahorros. Escriba un programa que comience leyendo la cantidad de
dinero depositada en la cuenta de ahorros, introducida por el usuario.
Después el programa debe calcular y mostrar por pantalla la cantidad
de ahorros tras el primer, segundo y tercer años. Redondear cada
cantidad a dos decimales.
\begin{verbatim}
dinero_depositado = int(input("Ingrese el dinero depositado:"))
primer_anio = dinero_depositado * 1.04
segundo_anio = primer_anio * 1.04
tercer_anio = segundi_anio * 1.04
print("Ahorros al finalizar el primer año:", round(primer_anio, 2))
print("Ahorros al finalizar el segundo año:", round(segundo_anio, 2))
print("Ahorros al finalizar el tercer año:", round(tercer_anio,2))
\end{verbatim}
\section{ejercicio 18:}
\label{sec:org26e174f}
Escriba un programa que permita ingresar una cantidad de
dinero c y un porcentaje p. El programa debe calcular el porcentaje p
de dinero de c.
\begin{verbatim}
dinero = float(input("Ingrese la cantidad de dinero:"))
porcentaje = float(input("Ingrese el porcentaje:"))
resultado = dinero * porcentaje / 100
print("El", porcentaje, "%", "de", dinero, "es:", round(resultado, 2))
\end{verbatim}
\end{document}