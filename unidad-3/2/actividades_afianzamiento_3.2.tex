% Created 2023-06-05 Mon 08:03
% Intended LaTeX compiler: pdflatex
\documentclass[11pt]{article}
\usepackage[utf8]{inputenc}
\usepackage[T1]{fontenc}
\usepackage{graphicx}
\usepackage{longtable}
\usepackage{wrapfig}
\usepackage{rotating}
\usepackage[normalem]{ulem}
\usepackage{amsmath}
\usepackage{amssymb}
\usepackage{capt-of}
\usepackage{hyperref}
\author{Juan Jose Della Vecchia}
\date{\today}
\title{Práctico Nro. 3.2: Tipos de Dato Colección - Listas}
\hypersetup{
 pdfauthor={Juan Jose Della Vecchia},
 pdftitle={Práctico Nro. 3.2: Tipos de Dato Colección - Listas},
 pdfkeywords={},
 pdfsubject={},
 pdfcreator={Emacs 28.2 (Org mode 9.5.5)}, 
 pdflang={English}}
\begin{document}

\maketitle
\tableofcontents

Nota: Asuma una cantidad especíca de elementos cuando el ejercicio no lo
especique.

\section{Ejercicio 1:}
\label{sec:org10421aa}
Dada la siguiente lista l=[10,"hola",2.5,20,"que",3.5,30,"tal",4.5]
se pide recuperar:
\begin{enumerate}
\item el 30
\item "hola"
\item 10,"hola",2.5
\item Los strings
\item Los flotantes
\item Los enteros
\end{enumerate}
\begin{verbatim}
l = [10,"hola",2.5,20,"que",3.5,30,"tal",4.5]
print (l[6])
print (l[2])
print (l[:3])
strings = "hola"
flotantes = 1.0
enteros = 1
print ("Los strings son:", strings)
print ("los flotantes son:", flotantes)
print ("los enteros son:", enteros)
\end{verbatim}

\section{Ejercicio 2:}
\label{sec:orgbb84a5f}
Realice las siguientes actividades:
\begin{enumerate}
\item Defina una lista l de tres números donde cada número es 0.
\item Defina una lista de un único elemento.
\item Defina una lista con n 0s.
\end{enumerate}
\section{Ejercicio 3:}
\label{sec:orgd6edd58}
Defina las listas l0 y l1 cada una con dos elementos numéricos

y luego construya la lista r cuyos elementos son la suma de los elemen-
tos de l0 y l1. Ejemplo: Si l0=[10,20] y l1=[8,20] la tupla r tiene que

contener r=[18,40].
\section{Ejercicio 4:}
\label{sec:orgea5178e}
Escriba un ejemplo que muestre que las listas son mutables.
\section{Ejercicio 5:}
\label{sec:orgd5e9e9f}
Escriba un programa que dada una lista t con 5 elementos y
un número n produzca como resultado una nueva lista con todos los
elementos de la lista t multiplicados por el número n.
\begin{verbatim}
l = [ 1, 1, 2, 3, 5]
n = 3
nueval = []
nueval = list(map (lambda el:el*n, l))
print(nueval)
\end{verbatim}

\section{Ejercicio 6:}
\label{sec:orgc642aa1}
Escriba un programa que almacene el valor de tres variables
ingresadas por el usuario en una lista.
\begin{verbatim}
lista = []
for x in range(3):
  n = input ("ingrese un elemento para agregar a la lista: ")
  lista.append(n)

print(lista)
\end{verbatim}
\section{Ejercicio 7:}
\label{sec:org785c6d5}
Escriba un programa que:
\begin{enumerate}
\item Permita que el usuario ingrese cuatro números, los almacene una
\end{enumerate}
lista l.
\begin{enumerate}
\item Genere una lista s la cual se obtiene sumando a cada elemento de
\end{enumerate}
l un valor ingresado por el usuario.
\begin{enumerate}
\item Genere una lista r la cual se obtiene restando a cada elemento de
\end{enumerate}
l un valor ingresado por el usuario.
\begin{enumerate}
\item Imprima: con leyendas adecuadas la tupla l, s y r.
\end{enumerate}
\begin{verbatim}
lista = []
for x in range(4):
  n = int(input ("ingrese un numero para agregar a la lista: "))
  lista.append(n)

print(lista)
\end{verbatim}
\section{Ejercicio 8:}
\label{sec:orga3b9a95}
Cree una lista y muestre:
\begin{enumerate}
\item El acceso a un elemento de la lista.
\item Qué sucede si se intenta acceder a una posición inexistente de la
\end{enumerate}
lista.
\begin{enumerate}
\item Cómo se calcula la longitud de una lista.
\end{enumerate}
\section{Ejercicio 9:}
\label{sec:org24a7ba5}
Construya un programa que permita que el usuario ingrese una
lista de dos elementos y luego desempaquete la lista en dos variables a
y b. Luego el programa debe imprimir las variables a y b.
\section{Ejercicio 10:}
\label{sec:orgd5582a3}
Escriba un programa que permita que el usuario ingrese dos
valores en las variables a y b y luego empaquete dichos valores en una
lista. Luego el programa debe imprimir la tupla resultado.
\section{Ejercicio 11:}
\label{sec:org455be60}
Escriba un programa que permita que el usuario ingrese un
número a y una lista l. Luego el programa debe imprimir True si el
número a está en l y False en otro caso.
\section{Ejercicio 12:}
\label{sec:orga519fc4}
Escriba un programa que permita que el usuario ingrese un
número a y una lista l. Luego el programa debe imprimir por pantalla
la posición del número a en la lista l. En caso de que el número a no se
encuentre en l el programa debe imprimir -1.
\section{Ejercicio 13:}
\label{sec:org43c934f}
Realice las siguientes actividades:
\begin{enumerate}
\item Explique el concepto de rodaja.
\item Explique el concepto de zancada.
\item Por cada concepto explicado de ejemplos.
\end{enumerate}
\section{Ejercicio 14:}
\label{sec:org71a819f}
Escriba un programa que permita que el usuario ingrese un
número a y una lista l. Luego el programa debe mostrar por pantalla
la cantidad de veces que aparece el número a en la lista l.
\section{Ejercicio 15:}
\label{sec:org599b37b}
Dada la lista l=[34, 3.2, Juan, Pedro,-2] se pide:
\begin{enumerate}
\item Agregue al nal de l un string ingresado por el usuario.
\item Solicite al usuario un elemento y cuente la cantidad de veces que
\end{enumerate}
aparece dicho elemento en l.
\begin{enumerate}
\item Pida al usuario una lista s e incorporela al nal de l.
\item Invierta la lista l.
\end{enumerate}
\section{Ejercicio 16:}
\label{sec:org9a67428}
Construya un programa que:
\begin{enumerate}
\item Permita que el usuario ingrese una lista l de números enteros l.
\item Ordene la lista
\item Almacene en la variable mayor el mayor elemento de la lista
\item Almacene en la variable menor el menor elemento de la lista.
\item Imprima por pantalla la lista l y el elemento mayor y el elemento
\end{enumerate}
menor.
\section{Ejercicio 17:}
\label{sec:org77de181}
Escriba un programa que:
\begin{enumerate}
\item Permita que el usuario ingrese una lista l.
\item Pida al usuario en elemento e.
\item Pida al usuario una posición p válida.
\item Inserte en la lista l el elemento e en la posición p.
\end{enumerate}
\end{document}